\documentclass[10pt,a4paper]{article}
\usepackage[latin3]{inputenc}
\usepackage{amsmath}
\usepackage{amsfonts}
\usepackage{amssymb}
\usepackage{graphicx}
\author{Ricardo Arango - Big Bang Data}
\title{Modelo de Deserci�n en Clientes}
\begin{document}
\maketitle
	\noindent La fuga de clientes se define como el \textit{Movimiento de un cliente de un proveedor de servicio a otro}, mientras que la gesti�n de fuga de clientes describe el proceso por el cual el operador del servicio intenta evitar la fuga del cliente. En consecuencia, la utilizaci�n de modelos predictivos que permitan \textbf{identificar clientes en situaci�n de riesgo de fuga}, permite al operador dirigir esfuerzos en prevenir la salida del cliente.
	\section{Los Datos}
		La variable \textit{target} corresponde a la variable objetivo que se quiere predecir. En este caso, el abandono voluntario de clientes, esta variable es de car�cter booleano, \textbf{FALSE: cliente permanece con sus servicios � TRUE: clientes los abandona voluntariamente}.
		\begin{figure}[h]
			\centering
			\includegraphics[width=0.5\textwidth]{count_churn}
		\end{figure}
\end{document}